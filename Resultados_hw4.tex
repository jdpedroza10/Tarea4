\documentclass{article}
\usepackage[utf8]{inputenc}

\title{RESULTADOS tarea 4 metodos computacionales}
\author{Julian Pedroza}
\date{19 de Noviembre}

\usepackage{graphicx}

\begin{document}

\maketitle

\section{Introduction}
En esta seccion se agregan las diferentes gráficas producidas en los puntos de ODE y PDE, mostrando el movimiento de un proyectil. 

\section{ODE}

En la primera parte se tomo como condicion inicial x(t) igual a 0 y una velocidad de 300 m/s y con un angulo de 45 grados.
Se probaron angulos de 10 a 70 grados y no los hice en una sola grafica por el codigo que utilice y ya no alcanzo a modificarlo pero aca muestro todos.
\begin{figure}[H]
\centering
\includegraphics[scale=0.7]{trayectoria_10.png}
\includegraphics[scale=0.7]{trayectoria_20.png}
\label{fig:Trayectorias}
\end{figure}

\begin{figure}[H]
\includegraphics[scale=0.7]{trayectoria_30.png}
\label{fig:Trayectorias}
\end{figure}


\begin{figure}[H]
\includegraphics[scale=0.7]{trayectoria_40.png}
\includegraphics[scale=0.7]{trayectoria_45.png}
\label{fig:Trayectorias}
\end{figure}

\begin{figure}[H]
\includegraphics[scale=0.7]{trayectoria_50.png}
\includegraphics[scale=0.7]{trayectoria_60.png}
\label{fig:Trayectorias}
\end{figure}

\begin{figure}[H]
\includegraphics[scale=0.7]{trayectoria_70.png}
\caption{Trayectorias}
\label{fig:Trayectorias}
\end{figure}

\section{PDE}
Evolucion temporal de la calcita hasta la configuracion de equilibrio. Incluye grafica de condiciones iniciales, dos graficas de estados intermedios y una grafica de configuracion de equilibrio.

\begin{figure}[H]
\includegraphics[scale=0.7]{caso_1_t0s.png}
\includegraphics[scale=0.7]{caso_1_t10000s.png}
\label{fig:Difusion termica}
\end{figure}

\begin{figure}[H]
\includegraphics[scale=0.7]{caso_1_t20000s.png}
\includegraphics[scale=0.7]{caso_1_t30000s.png}
\label{fig:Difusion termica}
\end{figure}

\begin{figure}[H]
\includegraphics[scale=0.7]{caso_2_t0s.png}
\includegraphics[scale=0.7]{caso_2_t10000s.png}
\label{fig:Difusion termica}
\end{figure}

\begin{figure}[H]
\includegraphics[scale=0.7]{caso_3_t20000s.png}
\includegraphics[scale=0.7]{caso_3_t30000s.png}
\label{fig:Difusion termica}
\end{figure}

\caption{Evolucion Difusion termica}

\section{Conclusion}
Some conclusion..


\end{document}
